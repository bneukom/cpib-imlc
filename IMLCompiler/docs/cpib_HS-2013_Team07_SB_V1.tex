\documentclass[a4paper,notitlepage,oneside]{article}

\title{Schlussbericht Compilerbau \\ Gruppe 7 \\ IML mit Listen}
\author{Christof Weibel \and Benjamin Neukom\\}
\date{15. November 2013\\ Version 1.0\\}
\usepackage{listings}
\usepackage{amsmath}
\usepackage{amssymb}
\usepackage{array}
\usepackage{txfonts} % ?
\usepackage{color}
\usepackage{mdwlist} % Für Listen mit * - kompakter
\usepackage{syntax}
\usepackage[T1]{fontenc} % bessere tilde und ^ in listintgs
\usepackage{needspace} %Zusammenhängende Text sequenzen definieren
\usepackage{amsmath}
\usepackage{fullpage}

\usepackage[utf8]{inputenc}

\definecolor{dkgreen}{rgb}{0,0.6,0}
\definecolor{gray}{rgb}{0.5,0.5,0.5}
\definecolor{mauve}{rgb}{0.58,0,0.82}
 
% "define" Scala
\lstdefinelanguage{scala}{
  morekeywords={abstract,case,catch,class,def,%
    do,else,extends,false,final,finally,%
    for,if,implicit,import,match,mixin,%
    new,null,object,override,package,%
    private,protected,requires,return,sealed,%
    super,this,throw,trait,true,try,%
    type,val,var,while,with,yield},
  otherkeywords={=>,<-,<\%,<:,>:,\#,@},
  sensitive=true,
  morecomment=[l]{//},
  morecomment=[n]{/*}{*/},
  morestring=[b]",
  morestring=[b]',
  morestring=[b]"""
}

\lstdefinelanguage{haskell}{
  morekeywords={True, False},
  otherkeywords={=>,<-,<\%,<:,>:,\#,@},
  sensitive=true,
  morecomment=[l]{//},
  morecomment=[n]{/*}{*/},
  morestring=[b]",
  morestring=[b]',
  morestring=[b]"""
}

\lstdefinelanguage{iml}{
  morekeywords={program, endprogram, global,int,bool,fun,in,copy,out,ref,inout,const,local,skip, endfun, returns, do,else,false, true, for,if,endif, init,endwhile,proc,endproc,init,debugin,debugout, val,var,while,head,tail,length},
  otherkeywords={=>,<-,<\%,<:,>:,\#,@},
  sensitive=true,
  morecomment=[l]{//},
  morecomment=[n]{/*}{*/},
  morestring=[b]",
  morestring=[b]',
  morestring=[b]"""
}

% Default settings for code listings
\lstset{frame=tb,
  language=scala,
  aboveskip=3mm,
  belowskip=3mm,
  showstringspaces=false,
  columns=flexible,
  basicstyle={\small\ttfamily},
  numbers=none,
  numberstyle=\tiny\color{gray},
  keywordstyle=\color{blue},
  commentstyle=\color{dkgreen},
  stringstyle=\color{mauve},
  frame=single,
  breaklines=true,
  breakatwhitespace=true
  tabsize=2,
  captionpos=b
}

\renewcommand{\syntleft}{\normalfont\itshape}
\renewcommand{\syntright}{}

\setlength{\grammarparsep}{0pt} % increase separation between rules
\setlength{\grammarindent}{16em} % increase separation between LHS/RHS 
\renewcommand{\grammarlabel}[2]{\hspace{10 em}{#1}\hfill#2}

\newcommand{\labracket}{[}
\newcommand{\rabracket}{]}
\newcommand{\inlinecode}[1]{\lstinline|#1|}

\begin{document}
\maketitle

\begin{abstract}
Dieses  Dokument beschreibt, wie die Sprache IML um Listen erweitert wird, welche Änderungen hierfür am Compiler nötig sind und wie das IML Programm nach Java Bytecode übersetzt wird. Die Spracherweiterung wird mit den Sprachen Scala und Haskell verglichen und die Designentscheide werden erläutert.  \\ \\
\end{abstract}

\section{Listen als Spracherweiterung}

Listen für IML sind wie folgt definiert

\begin{enumerate}

  \item Listen sind geordnete Ansammlungen von Objekten mit dem gleichen Datentyp $T$. Listen bestehen aus einem Head vom Typ $T$ und einem Tail einer Liste vom Typ $T$. Beispiel: $[1,2,3]$ ist eine Liste vom Typ $int$ mit den Werten 1, 2 und 3 (Head ist 1 und Tail ist $[2, 3]$). Oder mit Verschachtelung $[[true, false], [true]]$ ist eine Liste vom Typ Liste von bools mit den Werten $[true,false]$ und $[true]$.

  \item Stores vom Typ Liste können mit Brackets definiert werden. Beispiel: $var$ $l:[int]$ ist ein Store vom Typ Liste von ints.
    
  \item Der Typ einer Liste ist definiert durch das erste Element (falls das Element eine Liste ist, das erste nicht leere Element) der Liste. Beispiel:  $[[1,2,3],[3,4,5]]$ ist eine Liste vom Typ Liste von $int$s oder $[[], [true]]$ ist eine Liste vom Typ Liste von bools.

  \item Die leere Liste $[]$ ist vom Typ $Any$ (welcher in IML nicht verwendet werden kann). Der Typ $Any$ ist kompatibel mit allen anderen Typen. Beispiel: für die Deklaration des Stores $var$ $l:[[int]]$ ist $l := []$ (Typ von $[]$ ist $[Any]$ und $Any$ ist kompatibel mit $[int]$) so wie auch $l := [[]]$ (Typ von $[[]]$ ist $[[Any]]$ und $Any$ ist kompatibel mit $int$) eine gültige Zuweisung.

  \item Listen sind immutable, d.h. die Werte einer Liste können nicht verändert werden. Es können nur mit den Operationen $::$ (Cons) und $tail$ neue Listen konstruiert werden.
\end{enumerate} 

\newpage
\section{Operationen mit Listen}
\subsection{Cons}
\begin{align*}
Type :: [Type] \rightarrow [Type]
\end{align*}
Der $::$ Operator erstellt eine neue Liste mit dem Operand auf der linken Seite als Head und dem Operand der rechten Seite als Tail. Beispiel: die Expression $1 :: [2,3,4]$ gibt die Liste $[1,2,3,4]$ zurück. Der $::$ Operator ist rechtsassoziativ. Beispiel: die Expression 1 :: 2 :: 3 :: [] gibt die Liste $[1,2,3]$ zurück.

\subsection{Head}
\begin{align*}
head\;[Type] \rightarrow Type
\end{align*}
Der $head$ Operator gibt den Head einer Liste zurück. Beispiel: die Expression $head$ $[2,3,4]$ gibt den Wert 2 zurück. Falls versucht wird, den Head einer leeren Liste abzufragen, wirft die VM eine Exception.

\subsection{Tail}
\begin{align*}
tail\;[Type] \rightarrow [Type]
\end{align*}
Der $tail$ Operator gibt den Tail einer Liste zurück. Beispiel: die Expression $tail\;[1,2,3,4]$ gibt die Liste [2,3,4] zurück. Falls versucht wird, den Tail einer leeren Liste abzufragen, wirft die VM eine Exception.

\subsection{Length}
\begin{flalign*}
length\;[Type] \rightarrow Int
\end{flalign*}
Der $length$ Operator gibt die Länge einer Liste zurück. Beispiel: $length\;[1,2,3,4,5]$ gibt den Wert 5 zurück.

\section{List Comprehension}

\raggedright
Mit List Comprehensions können Listen erzeugt werden. Sie folgt der mathematischen Set-Builder Notation.


\begin{flalign*}
\{\;
\underbrace{\vphantom{f}3*x\;}_{Output\;Function} | 
\underbrace{\vphantom{f}x\;}_{Counter} 
\underbrace{from\;0\;to\;100}_{Range} \;when
\underbrace{\vphantom{f}\;x\;mod\;2\;==\;0}_{Predicate} 
\;\}
\end{flalign*}

\begin{description}
	\item[Output Function] \hfill \\
	Funktion welche auf die vom Predicate akzeptieren Elemente angewendet wird. Die Output Function muss einen Wert vom Typ $int$ zurückgeben.
	\item[Counter] \hfill \\
	Der Zähler, welche die angegebene Range durchläuft. Er ist  immer vom Typ $int$, deshalb muss dieser nicht explizit angegeben werden. Der zu dem Zähler gehörenden Store ist anonym, dass heisst, er kann nur innerhalb dieser List Comprehension verwendet werden.
	\item[Range] \hfill \\
	Die Inputmenge, angegeben durch zwei Expressions welche einen $int$ zurückgeben. Im Beispiel von 0 bis und {\bfseries mit} 100. 
	\item[Predicate] \hfill \\
	Filter, welcher auf jedes Element der Inputmenge angewendet wird, um zu entscheiden ob es in der resultierenden Liste enthalten ist. Die Filter Expression muss einen Wert vom Typ $bool$ zurückgeben.
	
\end{description}

Die List Comprehensions werden nach dem der Typechecker das Programm überprüft hat in While-Loops übersetzt. Somit müssen sie bei der Code Generierung nicht speziell behandelt werden.

\newpage
\section{Vergleich mit Haskell und Scala}
\subsection{Haskell}
Die Operatoren $head$, $tail$ und $length$ verhalten sich identisch zu den Haskell Varianten. Einen Unterschied zu Haskell ist die Präzedenz des Cons-Operators. Bei der IML Erweiterung hat der Operator :: die tiefste Priorität und bei Haskell liegt die Priorität zwischen den boolschen, logischen und den arithmetischen Operatoren. Dies hat folgende Konsequenzen:
\newline
\newline
Beispiel für Haskell: 

\begin{lstlisting}[language=haskell, caption=Ungültige Cons Operation in Haskell]
1 > 3 : True : [] // Type Error
(1 > (3 : (True : []))) // Mit Klammerung um Operator Praezedenz zu zeigen
\end{lstlisting}

\raggedright
Gleiches Beispiel in IML:

\begin{lstlisting}[language=iml, caption=Gültige Listen Konkatenation in IML]
1 > 3 :: true : [] // Keinen Type Error
((1 > 3) :: (true :: [])) // Mit Klammerung um Operator Praezedenz zu zeigen
\end{lstlisting}
Wir haben uns für diese Präzedenz entschieden, da es unserer Meinung nach natürlicher ist, dass der :: Operator die tiefste Priorität hat.
Ausserdem haben wir noch kein Beispiel gefunden, wo die Operator Präzedenz wie sie in Haskell implementiert ist, einen Vorteil gegenüber unserer Implementation bietet.

\subsection{Scala}
In Scala verhält sich die Operator Präzedenz des :: Operator gleich wie bei Haskell, also unterschiedlich zu IML.


\newpage
\section{Änderungen an der Grammatik}
Folgende Änderungen wurden an der Grammatik vorgenommen. Die Änderungen wurden mit einem eigens entwickelten Tool (mehr dazu später) getestet und sind $LL(1)$ konform.
\subsection{Expression}
\begin{grammar}
<expr> ::= <term0> \{CONCATOPR term0\}

<term0> ::= <term1> \{BOOLOPR term1\}

<term1> ::= <term2> [RELOPR term2]

<term2> ::= <term3> \{ADDOPR term3\}

<term3> ::= <factor> \{MULTOPR factor\}

<factor> ::= <literal>
		\alt IDENT [INIT | <exprList>]
		\alt monadicOpr <factor>
		\alt LPAREN <expr> RPAREN
		\alt listComprehension;
		
<exprList> ::= LPAREN [<expr> \{COMMA <expr>\}] RPAREN

<monadicOpr> ::= NOT | ADDOPR | HEAD | TAIL | SIZE

\end{grammar}


\subsection{List Comprehension}
Neu:
\begin{grammar}
<listComprehension> ::=  LCURL expr PIPE ident FROM expr TO expr WHEN expr RCURL
\end{grammar}

\subsection{Type}
Vorher:
\begin{grammar}
<atomType> ::= INT | BOOL
\end{grammar}
Nachher:
\begin{grammar}
<type> ::=  <atomType> | LBRACKET <type> RBRACKET

<atomType> ::= INT | BOOL
\end{grammar}

\subsection{Literal}
Vorher:
\begin{grammar}
<literal> ::= INTLITERAL | BOOLLITERAL
\end{grammar}
Nachher:
\begin{grammar}
<literal> ::= INTLITERAL | BOOLLITERAL | <listLiteral>

<listLiteral> ::= LBRACKET [<expr> \{COMMA <expr>\}] RBRACKET
\end{grammar}
\newpage

\section{Grammatik auf LL(1) Fehler überprüfen}
Die Grammatik wurde, mit einem eigens entwickelten Tool, auf LL(1) Fehler überprüft. Das Tool akzeptiert eine EBNF Grammatik (wie im Unterricht besprochen) und wandelt diese in eine normale Grammatik um. Danach werden die $NULLABLE$, $FIRST$ und $FOLLOW$ Sets berechnet und damit die Parse Tabelle generiert.
\newline
\newline
Grammatik der EBNF Grammatik:
\begin{grammar}
<grammar> ::= <production> \{SEMICOLON <production>\};

<production> ::= NTIDENT ASSIGN <term0>;

<term0> ::= \{<term1>\} \{PIPE \{<term1>\}\};

<term1> ::= <repTerm> | <optTerm> | <symbol>;

<repTerm> ::= LCURL <term0> RCURL;

<optTerm> ::= LBRAK <term0> RBRAK;

<symbol> ::= TIDENT | NTIDENT
\end{grammar}

\section{Code Generierung}
Da wir für Listen einen Heap benötigen haben wir uns entschieden für die Java Virtual Machine (JVM) zu kompilieren. Aus dem IML Programm wird Java Bytecode generiert.

\subsection{Aufbau}
\begin{description}
	\item[IML Programm] \hfill \\
	Wird in eine Java Klasse mit dem gleichen Namen 		übersetzt
	\item[Globale Felder] \hfill \\
	Werden in private statische Variabeln übersetzt
	\item[Methoden und Prozeduren] \hfill \\
	werden in statische Java Methoden übersetzt
	\item[IML Hauptprogramm] \hfill \\
	wird in die Java Main-Methode übersetzt
\end{description}

\subsection{Listen}
IML Listen werden in Java Arrays umgewandelt. Für die Operatoren $tail$ und $cons$ werden Java Methoden generiert um eine deep Copy durchzuführen, welche bei jedem Aufruf dieser Operatoren ausgeführt werden. Für den $head$ Operator wird keine deep Copy benötigt, da Listen immutable sind.

\subsection{Out Parameter}
Da die JVM keine Out oder Ref Parameter unterstützt müssen diese speziell behandelt werden. Out-Parameter wurden folgendermassen umgesetzt: Der übergebene Store wird in ein Array der Länge Eins gespeichert. In der Prozedur werden alle Lese und Schreibe Operationen über dieses Array ausgeführt. Nach dem Aufruf der Prozedur wird der Parameter wieder aus dem Array gelesen und in den eigentlichen Store geschrieben. Ref Parameter wurden bei unserer Implementation nicht speziell behandelt. Sie funktionieren gleich wie Copy Out Parameter.


\newpage
\section{Code Beispiele}
Summe der Elemente einer int Liste:
\begin{lstlisting}[language=iml, caption=Beispiel für die Berechnung der Summe der Element einer Liste in IML]
program listSum()
global
fun sum(in copy l:[int]) returns var r:int
do
	if length l == 0 do
		r init := 0
	else
		r init := head l + sum(tail l)
	endif
endfun;

var l:[int];
var sum:int
do
	l init := [1,2,3,4,5,6,7,8,9,10];
	sum init := sum(l);
	debugout sum
endprogram
\end{lstlisting}

List Contains:
\begin{lstlisting}[language=iml, caption=Listen Contains]
program listContains()
global

fun contains(in copy l:[int], in copy i:int) returns var r:bool
do
	if length l == 0 do
		r init := false
	else
		r init := head l == i || contains(tail l, i)
	endif
endfun;


var l:[int];
var i:int;
var c:bool
do
	l init := [1,2,3,4,5,6,7,8];
	debugout l;
	debugin i init;
	c init := contains(l,i);
	debugout c
endprogram

\end{lstlisting}

\newpage
Liste umkehren:
\begin{lstlisting}[language=iml, caption=Liste reverse 1]
program listReverse()
global

// returns last element
fun last(in copy l:[int]) returns var r:int
do
	if length l == 1 do
		r init := head l
	else
		r init := last(tail l)
	endif
endfun;

// init for haskell (list without last)
fun initial(in copy l:[int]) returns var  r:[int]
do
	if length l == 1 do
		r init := []
	else
		r init := head l :: initial(tail l)
	endif
endfun;

// would be easier with ++ operator
// reverses the given list and returns a new one
fun reverse(in copy l:[int]) returns var r:[int]
do
	if length l == 0 do
		r init := []
	else
		r init := last(l) :: reverse(initial(l)) 
	endif
endfun;


var l:[int];
var r:[int]
do
	l init := [1,2,3,4,5,6,7,8];
	r init := reverse(l)
endprogram
\end{lstlisting}

\newpage
Liste von Primzahlen mit List Comprehensions
\begin{lstlisting}[language=iml, caption=Primzahlen Liste]
program primesList() 
global 
	fun isPrime(in copy const p:int) returns var b:bool
	global
	local
		var c:int
	do
		c init := 2;
		
		if p > 1 do
			b init := true;
			while c < p do
				if p mod c == 0 do
					b := false
				else
					skip
				endif;
				c := c + 1
			endwhile
		else
			b init := false
		endif
	endfun;

	fun sum(in copy const l:[int]) returns var r:int
	local 
	var x:int
	do
		if length l == 0 do
			r init := 0
		else
			r init := head l + sum(tail l)
		endif
	endfun;

	var l:[int];
	var max:int
do 
	debugin max init;
	l init := { x | x from 0 to max when isPrime(x)};
	debugout l;
	debugout sum(l) 
endprogram
\end{lstlisting}

\newpage
\begin{lstlisting}[language=iml, caption=Liste reverse 2]
program listReverse2()
global

// reverse 2
fun reverse(in copy l:[int], in copy acc:[int]) returns var r:[int]
do
	if length l == 0 do
		r init := acc
	else
		r init := reverse(tail l, head l :: acc)
	endif
endfun;

var l:[int];
var r:[int]
do
	l init := [1,2,3,4,5,6,7,8];
	r init := reverse(l, [])
endprogram
\end{lstlisting}

\begin{lstlisting}[language=iml, caption=Teilbarkeit]
program divisibility() 
global 
	var l:[[int]];
	var max:int;
	var counter:int
do 
	debugin max init;
	
	l init := [];
	counter init := max;
	while counter > 0 do
		l := { x | x from 1 to max when x mod counter == 0} :: l;
		counter := counter - 1
	endwhile;
	
	debugout l
endprogram
\end{lstlisting}

\section{Quellen}
Internet:\\
\begin{tabular}{l l}
Wikipedia & en.wikipedia.org \\
Haskell Listen  & http://andres-loeh.de/haskell/4.pdf \\
Haskell Language Specification & www.haskell.org/onlinereport/ \\
Scala & www.scala.org
\end{tabular} \\
Bücher: \\
\begin{tabular}{l l}
Progranmming in Scala & Odersky et. al. \\
\end{tabular}

\end{document}


